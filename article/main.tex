\documentclass{article}
\usepackage[utf8]{inputenc}


\title{Etude d'un algorithme génétique sur le probleme one-max}
\author{BOKA Yao }
\date{Octobre 2018}

\begin{document}

\maketitle

\section{Introduction}

Les problemes d'optimisation peuvent etre divisés en deux grandes cathégories. On parle d'optimisation continue lorsque les variables manipulées sont continues et l'on parle d'optimisation combinatoire lorsque les variables manipulées sont discrètes.
Pour un probleme combinatoire donné, l'optimisation consiste à trouver une solution optimale dans un  ensemble de soutions possibles.
Les \emph{Algorithmes Evoluonnaires} font partie des methodes d'optimisation stochastiques, inspirés des processus biologiques et particulièrement des therories darwiniennes  selon lequelles un ensemble d'organismes biololgiques subissement des processus et pressions environnetaux qui leurs permettent de mieux s'adapeter à environnement. Il existe plusieurs variantes de ces algorithmes parmis lequelles on peut citer les \emph{strategies d'evolution}, la \emph{programmation évolutionnaire}, \emph{les algorithmes génetiques} et la \emph{programmation genetique}...
Le principe general d'un algorithme évolutionnaire est le suivant: On part d'un ensemble de solutions candidates genererées/obtenues aléatoirement representées par une popupations, on selectionne au fil des générations (représentées par un certains nombre d'itérations) des individus parents de la population sur lesquels sont appliqués les opérateurs de croisement et/ou de mutations. Le croisement est un opérateur qui consiste à combiner deux ou plusieurs individus parents afin d'obtenir un ou plusieurs nouveaux individus enfants. La mutation est une opérateur de modification qui s'applique sur un individu enfant. Les nouveaux individus enfants sont par la suite integrés dans la population par remplacement des individus déjà présents dans la population s'ils sont meilleurs, basé sur la \emph{fitness} (capacité d'adaptabilité) ou l'age.  Ce principe est représenté par le schéma dans la figure 1 ainsi que le pseudo code de l'algorithme dans la figure 2.
Les différentes variantes d'un algorithme révolutionnaire suivent ce même schéma mais diffèrent de l'espace de recherche (l'espace sur lequel est défini la fonction objectif), de la représentation du problème et de la définition des opérateurs de variations.

\section{Présentation de l'algorithme génétique et son application sur problème One-Max}

Les algorithmes génétiques (GA) ont été introduits en 1975 par John Holland et imaginés comme outils de modélisation de l'adaptation. On les caractérise principalement par la représentation du génotype de façon binaire (i.e un individu est représenté par une chaîne de bit).

\subsection{Les composants de l'algorithme}

Un individu est représenté comme une chaine de bits de taille N.
La fonction fitness f : permet de calculer le nombre de 1 que contient un individu. Exemple: avec S= 1100101, f(S) = 4
	
\paragraph{Initialisation:}
Plusieurs choix se presentent tels que partir d'une population de N individus dont tous les L bits constituant la chaine sont initalisés à 0, ou partir d'une population N individus da chaine bit est construite de façon aléatoire.
\paragraph{La selection:}
Dans le cadre de ce travail, 3 méthodes sont utilisées: 
La sélection best: qui consiste à choisir dans la population les deux meilleurs, (ie la meilleure évaluation selon f
La sélection aléatoire): qui consiste à choisir de façon aléatoire deux individus dans la population.
La sélection par tournoi : qui consiste à sélectionner T individus de façon aléatoire et d'en choisir les deux meilleurs

\paragraph{La recombinaison/croisement:}
\begin{itemize}
    \item \emph{Le Croisement mono-point:} Consiste à aléatoire un point de croisement, ce qui permet de croiser les parties des deux parents afin d'obtenir les deux individus enfants.
    \item \emph{Le croisement uniforme:} Consistent à tirer indépendemment ( avec p =0.5) pour chaque positions de quel parent proviendra le bit correspondant chez chaque enfant.
\end{itemize}

\paragraph{ La mutation:}
Les opérateurs de mutation permettent de modifier aléatoirement certains bits de chaque individus enfants. Le opérateurs de mutations étudiées sont 3:

\emph{La mutation Bit-Flip}: qui consiste à inverser chaque bits de l'individu selon une probabilité k.
-La mutation N-flip: consiste à tirer aléatoirement N positions de bits et d'inverser les bits correspondant.

\paragraph{Insertion:}
L'insertion consiste à insérer les individus enfants dans la nouvelle génération de population en fonction de certains critères. Dans ce travail, on a:

-L'insertion par Fitness: C'est à dire que les deux individus enfants remplacement les deux individus parents les moins bons ( fitness f)

-L'insertion par age: C'est à dire que les deux individus enfants remplacent les deux individus parents les plus âgés (i.e de la plus ancienne génération de population)

\subsection{Choix des opérateurs pour l'analyse de l'algorithme}
L'exécution l'algorithme est réalisée selon diverses configurations sous plusieurs co permettent d'analyser l'impact afin d'analyser l'impact de certains paramètres sur la performance de l'algorithme.

\begin{itemize}
    \item[-] Configuration pour l'étude des mutations: L'analyse des opérateurs de mutations consiste à évaluer les performances de l'algorithme en faisant varier les méthodes mutation et sur une configuration particulière choisie. La paramètres sont représentés dans le tableau ci dessous.
    
    \begin{tabular}{|c|c|c|c|c|c|c|}
    \hline
    sélection & insertion & pc & pm & taille & population & itérations \\
    \hline
    tournoi & âge & 0 & 1 & 100 & 20 & 1000\\
    \hline
    \end{tabular}
    
    \item[-] Configuration pour l'étude des probabilités de mutations:
    
    \begin{tabular}{|c|c|c|c|c|c|c|}
    \hline
    sélection & mutation & insertion & pc & taille & population & itérations \\
    \hline
    tournoi & bit-flip & âge & 0 & 200 & 20 & 4000\\
    \hline
    \end{tabular}
    
    
    \item[-] Configuration pour l'étude des croisements:
    
    \begin{tabular}{|c|c|c|c|c|c|c|c|}
     \hline
    sélection & mutation & insertion & pc & pm & taille & population & itérations \\
    \hline
    tournoi & 5-flip & âge & 1 & 0.5 & 200 & 20 & 2000\\
    \hline
    \end{tabular}
    
    \item[-] Configuration pour l'étude des probabilités de croisement
    
    
    \item[-] Configuration pour l'étude des insertions
    
    \item[-] Configuration pour l'étude des sélection
        
\end{itemize}

\subsection{Résultats}


\end{document}
